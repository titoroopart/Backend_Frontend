\section{[Code02] APIconsumer con java + gradle + Retrofit}

Una API (Aplication Programming Interface) es un una interfase que permitirá a los programas comunicarse compartiendo las funcionalidades, datos, etc. Esto facilita a los desarrolladores a disponer de muchas herramientas y base de datos disponibles y usarlos en sus aplicaciones ya sea conectándose a sus propios servidores o a servidores de terceros.
\\
Un ApiConsumer sera esa parte del código que nos permitirá conectarnos a una API del lado del servidor.\\
Para crear un API consumer necesitaremos lo siguiente:
\begin{enumerate}
  \item Elegir un lenguaje de programación en este caso Java21
  \item Una librería, En el caso de Java retrofit
  \item Configurar los http methods get, post, put, delete
  \item Agregar content type Authorization
  \item Definir el body en el caso de los métodos post put y delete
  \item Manejar la respuesta o el error de la petición
\end{enumerate}
En el caso de Java21 trabajaremos con la librería retrofit, para agregar esta libreria a un poryecto gradle groovy simplemente agregamos la dependencia a build.gradle

\begin{lstlisting}[language=java]
dependencies {
  \\ otras dependencias
    implementation 'com.squareup.retrofit2:retrofit:2.9.0'
    implementation 'com.squareup.retrofit2:converter-scalars:2.9.0'
}
\end{lstlisting}
Construimos un objeto retrofit basado en la siguiente documentación \url{https://square.github.io/retrofit/2.x/retrofit/}
\begin{lstlisting}[language=java]
// archivo app/src/main/java/ApiConnection.java
import retrofit2.Retrofit;
import retrofit2.converter.scalars.ScalarsConverterFactory;

public class ApiConnection {

  public static void connection(){
    Retrofit retrofit = new Retrofit.Builder()
      .baseUrl("https://api.example.com/")
      .addConverterFactory(ScalarsConverterFactory.create())
      .build();
  }
}
\end{lstlisting}
Con import importamos todas las librerías necesarias, luego con .Builder() a la hora de crear el objeto iniciamos el patrón de diseño builder para construir el objeto retrofit, los .baseUrl() y .addConverterFactory() son parámetros necesarios para construir el objeto y por ultimo .build() es la orden que construye el objeto retrofit.
\\
El metodo .addConverterFactory es importante para poder manejar el objeto serializado y desealizarlo en un string json etc.
\\
Para poder generar el request es necesario usar una interface para el endpoint, esta interfase sera capas de manejar las peticiones get, post, put, delete etc y recibir variables si así lo requiere ej endpoint random:
\begin{lstlisting}[language=java]
import retrofit2.http.GET;
import retrofit2.Call;

  public interface getConnection {
    @GET("random")
    Call<String> getPost();
}
\end{lstlisting}
Ejemplo interface con parámetro id en el endpoint.
\begin{lstlisting}[language=java]
import retrofit2.http.GET;
import retrofit2.http.Path;
import retrofit2.Call;

  private interface GetList {
      @GET("posts/{id}")
      Call<String> getPost(@Path("id") int id); // Retorna String en lugar de objeto
  }
\end{lstlisting}

