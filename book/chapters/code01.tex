\section{[Code01]Instalación de Java JDK y Gradle}
\subsection{Instalación de Java JDK Linux}
\subsubsection{Instalación de Java JDK}
Para instalar java jdk en linux se debe seguir los siguientes pasos:
Instalamos Java JDK
\enablenumbering
\begin{lstlisting}[language=bash]
sudo apt update
sudo apt install default-jdk
\end{lstlisting}
\subsubsection{Configuramos las variables de entorno}
Primero buscamos la ruta exacta de tu JDK

\begin{lstlisting}[language=bash]
ls /usr/lib/jvm/
\end{lstlisting}
Busca el folder exacto de tu JDK (ej: jdk-21-amd64) y agrega la ruta a bashrc con los siguientes comandos.
\begin{lstlisting}[language=bash]
echo 'export JAVA_HOME=/usr/lib/jvm/jdk-21-amd64/' >> ~/.bashrc
echo 'export PATH=$JAVA_HOME/bin:$PATH' >> ~/.bashrc
source ~/.bashrc
\end{lstlisting}
Verificar instalación de Java
\begin{lstlisting}[language=bash]
java -version
javac -version
\end{lstlisting}

\subsubsection{Instalar Gradle}
Con bash
\begin{lstlisting}[language=bash]
curl -s "https://get.sdkman.io" | bash
source "\$HOME/.sdkman/bin/sdkman-init.sh"

sdk install gradle

sdk list

sdk use gradle 8.14
exit
\end{lstlisting}
Correr esto solo si usas fishshell
\begin{lstlisting}[language=bash]
fisher install reitzig/sdkman-for-fish
\end{lstlisting}

Crear una proyecto Gradle
\begin{lstlisting}[language=bash]
mkdir mi-proyecto
cd mi-proyecto
gradle init
\end{lstlisting}
Creamos el proyecto java aplicación y proyecto simple con groovy
Dentro de mi-proyecto deberías ver los siguientes archivos\\
\begin{enumerate}
  \item build.gradle: Archivo de configuración principal de gradle
  \item settings.gradle: Archivo donde se define la configuración del proyecto
  \item src/ Contiene el código fuente y los tests
\end{enumerate}

\subsection{Instalación de Java JDK y Gradle en Windows}
\subsubsection{Descarga e Instala Java JDK}
Descarga Java de su pagina oficial buscando java jdk en google, \url{https://www.oracle.com/java/technologies/downloads/}  para instalar Java JDK Ejecuta el instalador y sigue los pasos (usa la ruta por defecto, ej: 
\begin{lstlisting}[language=bash]
C:\Program Files\Java\jdk-21).
\end{lstlisting}
\subsubsection{Configurar Variables de Entorno} 
\begin{enumerate}
  \item Abre Editar variables de entorno del sistema (busca en el menú Inicio).
  \item En Variables del sistema, haz clic en Nueva y agrega: Nombre: JAVA\_HOME
\\
\begin{lstlisting}[language=bash]
Valor: C:\Program Files\Java\jdk-21 \$(ajusta la version).
\end{lstlisting}
\item Edita la variable Path y agrega:
\begin{lstlisting}[language=bash]
\%JAVA\_HOME\%\bin
\end{lstlisting}
\end{enumerate}
Verificar Java abriendo CMD y ejecuta:
\begin{lstlisting}[language=bash]
java -version
javac -version
\end{lstlisting}
(Debe mostrar la versión instalada).
\\
\subsubsection{Instalar Gradle}
 Descarga Gradle desde \url{https://gradle.org/install/#manually} (versión "Binary-only"). Extrae el ZIP en la siguiente ruta crear carpeta Gradle en el disco C:
\begin{lstlisting}[language=bash]
 C:\Gradle\gradle-8.14
\end{lstlisting}
Configurar Variables de Entorno para Gradle\\
En "Variables del sistema", Edita Path y agrega:
\begin{lstlisting}[language=bash]
C:\Gradle\gradle-8.14\bin
\end{lstlisting}